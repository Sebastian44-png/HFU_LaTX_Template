%!TEX root = ../main.tex

\chapter{Einleitung}\label{chap:Einleitung}

\section{Blind Text}

\blindtext[5]

\section{Beispiel Tabelle}

\begin{table}[H]
	\centering
	\begin{tabular}{c|c|c}
		\hline
		\textbf{Head1} & \textbf{Head2} & \textbf{Head3} \\ 
		\hline
		\hline 
		
		Val1 & Val2 & Val3 \\
		Val4 & Val5 & Val6 \\
		Val7 & Val8 & Val9 \\
		
		\hline
	\end{tabular} 
	\caption{Besipielüberschrift}
	\label{tab:Qualitätswerte}
\end{table}

\section{Beispiel Pseudocode}

\begin{algorithm}
	\caption{Bubble sort}
	\label{alg:bubblesort}
	\begin{algorithmic}[1]
		\STATE{\textbf{input}: Nicht sortierte Liste A[]}
		\STATE{\textbf{output}: Sortiertes A[]}
        \STATE{N = length(A)}
        \FOR{j = 1 to N}
			\FOR{i = 0 to N-1}
				\IF{A[i] > A[i+1]}
				\STATE{temp = A[i]}
				\STATE{A[i] = A[i+1]}
				\STATE{A[i+1] = temp}
				\ENDIF
			\ENDFOR
        \ENDFOR
	\end{algorithmic}
\end{algorithm}

\section{Besipiel für eine Formel mit Konditionen}

\begin{equation*}
	E = mc^2
\end{equation*}
\begin{conditions}
	E	&	Energie \\
	m	&	Masse \\
	c	&	Lichtgeschwindigkeit  
\end{conditions}
